The work presented in this thesis has been carried out within the framework of the ATLAS experiment at the CERN Large Hadron Collider (LHC), during my doctoral studies at the Instituto de Física Corpuscular (IFIC, CSIC–Universitat de València). Its aim is to advance both the understanding of the Standard Model and the development of experimental techniques that enable precision measurements of the Higgs boson sector.

The thesis is structured around two complementary lines of research. The first concerns the identification of electrons in the ATLAS detector. This project began as part of my qualification task in the collaboration, with the dual purpose of becoming an ATLAS author and gaining familiarity with the ATLAS software environment and working dynamics. As the project evolved, it proved highly promising and became a major focus of my doctoral research. This led me to devote increasing effort to the development of new strategies for electron identification, in particular through the use of machine learning techniques. The work culminated in the refinement, training, and optimisation of a deep neural network for electron identification, studied in detail against the likelihood-based method traditionally employed by ATLAS. This effort not only resulted in improved performance in terms of background rejection, but also fostered my broader interest in the role of advanced machine learning in high energy physics analyses.

The second line of research is devoted to the study of Higgs boson production in association with top quarks through the $H\to\tau\tau$ decay channel. The $t\bar{t}H$ process provides direct sensitivity to the top Yukawa coupling, the strongest fermionic interaction in the Standard Model, and plays a central role in testing the dynamics of electroweak symmetry breaking and the stability of the Higgs potential. The analysis was originally based on the full Run-2 dataset collected by ATLAS in proton-proton collisions at a centre-of-mass energy of $\sqrt{s} = 13$~TeV, corresponding to an integrated luminosity of $139~\mathrm{fb}^{-1}$. It employs a multivariate strategy to discriminate signal from the overwhelming backgrounds, and the results are interpreted in the Simplified Template Cross-Section (STXS) framework, allowing for measurements in dedicated kinematic regions and providing input for global Higgs coupling fits. Furthermore, an extension of this analysis is presented also including the early Run-3 dataset (2022–2024), which coincides with the years of my doctoral studiesm combined to the reprocessed Run-2 data with a new ATLAS software release, corresponding to approximately $166~\mathrm{fb}^{-1}$ of $pp$ collisions at $\sqrt{s} = 13.6$~TeV plus $140~\mathrm{fb}^{-1}$ at $\sqrt{s} = 13.6$~TeV. In this context, in addition to the $t\bar{t}H$ process, the associated production of a Higgs boson with a single top quark ($tH$) is also studied, for which the $H \to \tau\tau$ channel shows promising sensitivity.

The thesis is organised as follows. Chapter~1 reviews the theoretical framework of the Standard Model and the Higgs mechanism, with emphasis on the phenomenology of the top quark and the Higgs boson at the LHC. Chapter~2 introduces the LHC and the ATLAS detector, while Chapter~3 describes the data and simulated samples used in the analyses. The reconstruction of physics objects is detailed in Chapter~4. Chapter~5 presents an overview of the machine learning techniques employed, followed in Chapter~6 by the development of the deep neural network for electron identification and its performance evaluation. Chapter~7 is devoted to the $t\bar{t}H$ analysis in the $H\to\tau\tau$ channel during Run-2, including the event selection, background estimation, systematic uncertainties, and final results. Chapter~8 extends this analysis to the early Run-3 dataset and incorporates, for the first time, a study of the $tH$ production mode in $H \to \tau\tau$ decays. Finally, Chapter~9 summarises the main conclusions and perspectives.

This thesis reflects both methodological developments in particle identification and their application to precision Higgs boson measurements, highlighting the interplay between advanced analysis techniques and the exploration of fundamental physics at the LHC.
