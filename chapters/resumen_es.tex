
\chapter*{Resumen en español}
\addcontentsline{toc}{chapter}{Resumen en español}

Este documento presenta de manera extensa y detallada el trabajo de investigación realizado en el marco
del experimento ATLAS del CERN, dentro del Gran Colisionador de Hadrones (LHC). A lo largo de la tesis se han
abordado dos líneas de investigación principales y complementarias: por un lado, el desarrollo de nuevas técnicas
de identificación de electrones mediante redes neuronales profundas, y por otro, el estudio de la producción del
bosón de Higgs en asociación con quarks top en el canal de decaimiento a pares de leptones tau. Este resumen en
castellano sintetiza los resultados obtenidos en todo el documento, ofreciendo una visión global del contexto
teórico, el marco experimental, los desarrollos metodológicos y las contribuciones fenomenológicas. 

% ===========================
% INTRODUCCIÓN Y MOTIVACIÓN
% ===========================
La física de partículas moderna se apoya en el Modelo Estándar (SM), una teoría cuántica de campos que describe
con gran éxito tres de las cuatro interacciones fundamentales conocidas: la fuerte, la débil y la electromagnética.
A pesar de su carácter predictivo, el SM deja sin respuesta cuestiones esenciales como la naturaleza de la materia
oscura, la jerarquía de masas de las partículas elementales, el origen de la asimetría materia-antimateria o la
inclusión de la gravedad en un marco cuántico. En este contexto, el bosón de Higgs, descubierto en 2012 por ATLAS
y CMS, constituye la piedra angular que confiere masa a los fermiones y bosones portadores de la interacción débil.
Su estudio en detalle es crucial no solo para verificar la estructura del SM, sino también para abrir posibles
ventanas hacia nueva física.

El quark top, la partícula elemental más masiva conocida, desempeña un papel igualmente central. Su fuerte
acoplamiento de Yukawa con el bosón de Higgs lo convierte en un actor fundamental en la estabilidad del vacío
electrodébil y en la dinámica de ruptura espontánea de simetría. Precisamente el proceso de producción de Higgs en
asociación con un par de quarks top (\ttH) permite acceder directamente a esta interacción, convirtiéndose en uno
de los canales más relevantes y desafiantes de la física del LHC.

La tesis doctoral se estructura en torno a estos dos ejes. El primero tiene un marcado carácter metodológico y se
centra en la identificación de electrones en ATLAS. A través del desarrollo, entrenamiento y optimización de un
modelo basado en redes neuronales profundas, se logra una mejora significativa respecto al método clásico basado
en verosimilitud. El segundo eje es de carácter fenomenológico y aborda el estudio del proceso \ttH en el canal
$H\to\tau\tau$, empleando técnicas multivariantes avanzadas y analizando tanto los datos de Run-2 como los
primeros de Run-3.

% ===========================
% MARCO TEÓRICO
% ===========================
El documento arranca con una revisión del marco teórico. El Modelo Estándar describe fermiones organizados en tres
familias, los bosones portadores de interacción y el mecanismo de Brout--Englert--Higgs, mediante el cual las
partículas adquieren masa sin romper la invariancia gauge. Se detallan las propiedades del bosón de Higgs, sus modos
de producción en el LHC (fusión de gluones, fusión de bosones vectoriales, producción asociada a bosones W/Z y
asociada a quarks top), y sus principales canales de decaimiento (a quarks bottom, leptones tau, fotones y bosones W/Z).

\begin{figure}[htbp]
  \centering
  \includegraphics[width=0.7\textwidth]{images/higgs_prod_decay.pdf}
  \caption{Modos dominantes de producción y decaimiento del bosón de Higgs en colisiones protón-protón.}
  \label{fig:higgs_resumen}
\end{figure}

Se hace especial hincapié en la relevancia del acoplamiento de Yukawa con el quark top, cuyo valor cercano a la
unidad lo convierte en la interacción fermiónica más intensa del SM. El análisis de este acoplamiento, a través del
proceso \ttH, permite probar la estructura del potencial de Higgs y explorar la estabilidad del vacío hasta escalas
energéticas cercanas a la de Planck.

% ===========================
% LHC Y ATLAS
% ===========================
En la segunda parte se describe el marco experimental: el Gran Colisionador de Hadrones (LHC), la máquina más
potente jamás construida, y el detector ATLAS, uno de los cuatro grandes experimentos del LHC. Se explican las
características principales del acelerador, como la energía en el centro de masas (13 y 13.6 TeV en Run-2 y Run-3),
la luminosidad alcanzada y los retos derivados del fenómeno conocido como \textit{pile-up}, que consiste en múltiples
colisiones protón-protón en un mismo cruce de haces.

El detector ATLAS se detalla en sus distintos subsistemas: el trazador interno, encargado de reconstruir trayectorias
de partículas cargadas; los calorímetros electromagnéticos y hadrónicos, que miden la energía depositada; el
espectrómetro de muones, que identifica y mide partículas penetrantes; y el sistema de disparo y adquisición de
datos. El calorímetro electromagnético de argón líquido destaca por su papel crucial en la reconstrucción de
electrones y fotones, siendo protagonista en esta tesis.

\begin{figure}[htbp]
  \centering
  \includegraphics[width=0.75\textwidth]{images/atlas_detector.pdf}
  \caption{Esquema general del detector ATLAS, mostrando sus principales subsistemas.}
  \label{fig:atlas_resumen}
\end{figure}

% ===========================
% SIMULACIONES Y MUESTRAS
% ===========================
El trabajo dedica también una sección a describir la generación de muestras simuladas mediante programas de Monte
Carlo, esenciales para entrenar algoritmos de identificación y estimar fondos en los análisis. Se explican los
generadores empleados, la simulación de la respuesta del detector y la importancia de disponer de muestras
equilibradas y realistas de electrones de señal y procesos de fondo como $Z\to\tau\tau$, $t\bar{t}$ y multijet.

% ===========================
% TÉCNICAS DE APRENDIZAJE AUTOMÁTICO
% ===========================
El quinto capítulo introduce las técnicas de aprendizaje automático utilizadas en la tesis. Se presentan conceptos
básicos de redes neuronales profundas: capas densas, funciones de activación, regularización, algoritmos de
optimización y procedimientos de entrenamiento. Se explica cómo estos modelos pueden aprender relaciones
complejas y no lineales entre decenas de variables experimentales, mejorando la discriminación entre señal y fondo
respecto a métodos más tradicionales. También se describe brevemente el uso de bosques de decisión potenciados
(BDTs), empleados en el análisis de \ttH.

% ===========================
% RECONSTRUCCIÓN E IDENTIFICACIÓN DE ELECTRONES
% ===========================
El sexto capítulo constituye uno de los pilares de la tesis: la reconstrucción e identificación de electrones. Se
explica el proceso de reconstrucción, desde la formación de clústeres en el calorímetro hasta el emparejamiento con
trazas en el detector interno. Posteriormente se presentan los algoritmos de identificación: el método clásico
basado en verosimilitud (LH) y el nuevo enfoque basado en redes neuronales profundas (DNN).

La DNN fue entrenada con muestras etiquetadas de electrones de señal y diferentes tipos de fondo: jets ligeros,
fotones convertidos, hadrones pesados y leptones no puntuales. Las variables de entrada incluyen formas de los
depósitos de energía en el calorímetro, información de las trazas y variables globales del evento. Se aplicaron
técnicas de preprocesado y corrección de variables para asegurar un entrenamiento robusto.

\begin{figure}[htbp]
  \centering
  \includegraphics[width=0.7\textwidth]{images/dnn_vs_lh.pdf}
  \caption{Comparación entre el rendimiento del discriminante basado en DNN y el método de verosimilitud (LH).}
  \label{fig:dnn_lh_resumen}
\end{figure}

Los resultados muestran una mejora significativa en el rechazo de fondos para una misma eficiencia de señal. Se
definieron diferentes \textit{working points} (loose, medium, tight) optimizando la relación entre eficiencia y
pureza, lo que permitirá su aplicación flexible en futuros análisis de ATLAS.

% ===========================
% EFICIENCIAS Y MEDIDAS
% ===========================
Una parte importante del trabajo consistió en validar experimentalmente las eficiencias del nuevo algoritmo mediante
el método de \textit{tag-and-probe} en eventos $Z\to ee$. Se obtuvieron curvas de eficiencia en función de la energía
transversal y pseudorapidez del electrón, confirmando la mejora observada en simulación. Estas medidas garantizan
que el nuevo identificador pueda emplearse con confianza en análisis de física de precisión.

% ===========================
% ttH EN EL CANAL H->TAUTAU
% ===========================
El séptimo capítulo aborda la segunda línea de investigación: el análisis del proceso \ttH en el canal $H\to\tau\tau$.
Este canal, aunque experimentalmente complejo debido a la presencia de múltiples neutrinos en los decaimientos de los
taus, ofrece una oportunidad única para medir directamente el acoplamiento de Yukawa del quark top.

El análisis se llevó a cabo sobre el conjunto completo de datos de Run-2 (139 fb$^{-1}$ a $\sqrt{s}=13$ TeV). Se definieron
criterios de selección de objetos y desencadenadores específicos, se implementaron técnicas de reconstrucción de la masa
del sistema di-tau y se emplearon BDTs multiclase para separar la señal de los principales procesos de fondo: $t\bar{t}$,
$Z\to\tau\tau$ y producción asociada de Higgs con un solo top (tH).

\begin{figure}[htbp]
  \centering
  \includegraphics[width=0.75\textwidth]{images/postfit_tth_tautau.pdf}
  \caption{Distribuciones post-fit en regiones de señal del análisis de \ttH en el canal $H\to\tau\tau$.}
  \label{fig:tth_resumen}
\end{figure}

Los resultados se interpretaron en el marco del esquema de secciones eficaces simplificadas (STXS), lo que permite
realizar comparaciones directas con predicciones teóricas y con otros canales de producción y decaimiento del Higgs.

% ===========================
% RUN-3
% ===========================
\todo{Completar con los resultados preliminares del análisis de \ttH en Run-3 y el estudio de la producción de tH.}

% ===========================
% CONCLUSIONES
% ===========================
En las conclusiones se destaca la doble aportación de esta tesis. En el ámbito metodológico, el desarrollo de un
nuevo identificador de electrones mediante DNN constituye un avance sustancial respecto al método previo, con
impacto inmediato en la mejora de la pureza de muestras electrónicas en ATLAS. En el ámbito fenomenológico, el
análisis de \ttH en el canal $H\to\tau\tau$ aporta una medida directa del acoplamiento de Yukawa del top, clave para
la estabilidad del vacío electrodébil y para futuras combinaciones globales de acoplamientos del Higgs.

Este resumen extenso refleja la importancia de integrar técnicas avanzadas de aprendizaje automático en la física
experimental de partículas, y cómo estas herramientas permiten extraer información más precisa de los complejos datos
generados en el LHC. El trabajo combina innovación metodológica y análisis fenomenológico de frontera, contribuyendo
al programa científico de ATLAS y al conocimiento global de las propiedades del bosón de Higgs.

