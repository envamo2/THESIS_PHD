The work developed in this thesis illustrates different stages of the ATLAS experiment workflow, from the development of improved identification algorithms for basic physics objects, to the implementation of advanced analyses targeting specific Higgs production processes. In this sense, the topics addressed provide a broader picture of the multiple layers of experimental effort required to achieve high-precision measurements and extend the sensitivity of the LHC physics programme.

The studies on electron identification stand as a representative example of the preliminary object-performance work that strengthen the success of physics analyses. The transition from the likelihood-based method to a deep neural network marks a significant step forward, providing more powerful discrimination between signal and background electrons. This type of work ensures that subsequent analyses, such as \htautau or other Higgs channels, can rely on high-quality, well-calibrated inputs when deriving their final results. It also exemplifies the continuous effort within ATLAS to modernise its reconstruction and identification techniques in view of the increasing challenges posed by higher luminosity and more complex data-taking conditions.

On the physics-analysis side, the thesis has contributed to two complementary directions. The first is the study of Higgs production in association with a top-antitop pair decaying to \taul using the full Run-2 dataset, in the fully hadronic channel, embedded within the global \htautau analysis. This channel, though experimentally challenging, provides a direct probe of the top Yukawa coupling through one of the rarest accessible Higgs decay modes. The second is the combined study of \thqb and \ttH production using the full Run-2 dataset together with partial Run-3 data recorded to date (2022-2024). In this case, the focus has been placed on evaluating the achievable sensitivity to \thqb production when measured simultaneously with \tth in the \tauhadhad final state, a novel channel in which single-top–associated Higgs production has never been explored before. The results obtained represent a first step towards constraining this process in a new topology, opening the path for more sensitive measurements with the increasing Run-3 dataset.

The following sections summarise the main conclusions of these three research lines, highlighting the impact of the results achieved within the scope of Run-2 and the ongoing Run-3 programme.

\section{Electron identification using a DNN}
The first major contribution of this thesis has been the development and implementation of a deep neural network for electron identification in ATLAS. This represents a methodological shift with respect to the likelihood-based (LH) approach employed during Run-2, introducing a more flexible and powerful discriminant capable of capturing non-linear correlations between input variables. The use of a multi-layer network trained on a large set of high-level variables has led to a consistent improvement in background rejection while maintaining high signal efficiency.

The results obtained demonstrate clear gains with respect to the LH method across the full phase space. For electrons in the central barrel region, the DNN achieves an improvement in background rejection of up to 30–40\% at fixed signal efficiency, particularly in the rejection of fake electrons from photon conversions and jets misidentified as electrons. In the crack regions, where the traditional LH method shows reduced performance due to the more complex detector environment, the DNN yields even larger gains, with rejection factors up to 50\% higher in the range 
$1.37 < \eta <1.52$. The performance is also stable across transverse momentum, with the most significant improvements observed at low and intermediate \et ($20 <$ \et $< 40$~GeV), which are especially relevant for Higgs and electroweak measurements.

The DNN approach had been already benchmarked against the LH discriminant in terms of efficiency versus background rejection curves and ROC curves representation. In all cases, the DNN exhibits superior separation power. At working points corresponding to 80\% signal efficiency, the rejection of heavy-flavour jets is increased by approximately 25\%, while the rejection of light-flavour jets and photon conversions is improved by 30–40\%. These gains translate into more robust electron selections for analyses where background suppression is critical, such as Higgs boson decays to leptons.

Beyond its direct performance improvements, the DNN also provides a more flexible framework for defining working points. The continuous nature of its discriminant output allows for the optimisation of cuts tailored to specific physics analyses, offering finer control over the trade-off between efficiency and purity. A clear example is provided by the treatment of conversion fakes (CF): in the LH approach their rejection relied on the use of additional external discriminants (ECIDs), which are no longer available in release 22. With the DNN, however, the main discriminant combined with a dedicated CF discriminant recovers and even surpasses the performance of the LH+ECID strategy, demonstrating that the new framework can incorporate and improve functionalities that previously required separate tools. This approach is not limited to CF electrons, but can naturally be extended to other classes of background, enabling further performance gains while simplifying the overall strategy. 

In summary, the transition from the LH to the DNN represents a significant step forward in electron identification in ATLAS. The observed improvements in background rejection, directly benefit the sensitivity of precision measurements and searches relying on clean electron signatures. The next steps will involve the validation of the method in Run-3 data, the derivation of scale factors in dedicated control samples, and its integration as the default electron identification algorithm in ATLAS analyses.

\section{Measurement of \ttH with \htautau at $\sqrt{s}=13$~GeV}

The second main contribution of this thesis has been the study of the Higgs boson production in association with a top–antitop pair in the final state where the Higgs decays to a pair of \taul. This process provides one of the most direct probes of the top Yukawa coupling, since the production mode itself arises at leading order from the interaction of the Higgs boson with the top quark. The \htautau decay mode, in turn, offers sensitivity to the Yukawa coupling to leptons of the third generation, making this channel unique in its ability to test simultaneously two of the fundamental Higgs–fermion interactions predicted by the Standard Model.

Within ATLAS, the \(\ttH(\tau\tau)\) channel had already been studied in an earlier Run-2 analysis, although with limited sensitivity and relatively large uncertainties. In the most recent analysis presented here, based on the full Run-2 dataset, the strategy was significantly improved by employing more refined multivariate techniques and by extending the measurement to a larger number of STXS bins. The work presented in this thesis has contributed to this development, in particular to the optimisation of the event categorisation and of the discriminants used to separate signal from the dominant backgrounds. A dedicated multiclass training against the challenging \(\,Z\to\tau\tau\,+\)jets and \(\ttbar\) processes was carried out, leading to a notable gain in sensitivity in the extraction of the \(\ttH\) signal strength in the \(\tau_{\mathrm{had}}\tau_{\mathrm{had}}\) final state.  

The results obtained show that, despite the intrinsic experimental challenges of this channel, the analysis provides meaningful constraints on the \(\ttH\) production cross-section. The measured signal strength in the \(\tau\tau\) channel, \(\mu_{\ttH}^{\tau\tau}\), was consistent with the Standard Model expectation within uncertainties. The relative uncertainty was reduced by nearly a 18\%, representing a considerable improvement in the precision of this measurement. Furthermore, for the first time in this final state, results were extracted in three dedicated \(p_{\mathrm{T}}^{H}\) STXS bins: below 120~GeV, between 200 and 300~GeV, and above 300~GeV. Due to the limited statistical power, upper limits were also derived in the cross-sections in these three POIs, which represent the first ATLAS constraints on \(\ttH(\tau\tau)\) production in the STXS framework.  

When compared to CMS, which also reported results in \(\ttH(\tau\tau)\) using the full Run-2 dataset, the ATLAS analysis achieved a competitive sensitivity. Both experiments obtained measurements compatible with the Standard Model within uncertainties, reinforcing the robustness of the LHC-wide experimental effort.

Nevertheless, the analysis remains statistically limited, with the relatively small branching fraction of \htautau and the large backgrounds from top-quark production posing significant challenges. Systematic uncertainties, in particular those associated with the theoretical modelling of \ttbar and our signal, also impact the sensitivity. Despite these limitations, the analysis demonstrates that the \tauhadhad final state can be successfully exploited in \ttH searches and provides an essential cross-check to more abundant Higgs decay channels.

In summary, the \(\ttH(\tau\tau)\) analysis with Run-2 data has demonstrated that, despite the limited statistics available for this process, it is possible to achieve an improved sensitivity compared to previous iterations. The results are consistent with the Standard Model prediction and competitive with those obtained by CMS. Further improvements are expected as additional data are incorporated, particularly from Run-3, which will enable more precise measurements. In addition, a new ATLAS combination of this analysis with other Higgs decay channels and production modes is currently in preparation. This will allow us to assess the extent to which the inclusion of the \(\ttH(\tau\tau)\) channel reduces the anti-correlation with other Higgs final states and improves the overall determination of the top-quark Yukawa coupling.  

\section{\thqb + \ttH analysis with \htautau using Run-2 and partial Run-3 data}

The third and final main contribution of this thesis has been the combined study of Higgs boson production in association with a single top quark (\(tHqb\)) and with a top-antitop pair (\(\ttH\)). While \(\ttH\) production has already been firmly established at the LHC, the \(tH\) process remains far less explored. Its cross-section is about an order of magnitude smaller, and it features a unique interference pattern between diagrams proportional to the top-quark Yukawa coupling and those involving the Higgs-$W$ couplings. This makes \(tH\) particularly sensitive not only to the magnitude but also to the relative sign of the top Yukawa coupling.  

In this thesis, the first exploration of \(tHqb\) production in the fully hadronic \(\tau\tau\) final state has been carried out, in combination with \(\ttH\), using the full Run-2 dataset together with the partial Run-3 data recorded to date (2022-2024). This represents the first time that single-top-associated Higgs production is probed in a final state without prompt leptons, a topology that had never been targeted before at ATLAS. The analysis required a sophisticated categorisation strategy based on jet and \(b\)-tagged jets multiplicities, as well as the development of a new dedicated multiclass BDT discriminant to distinguish signal both signals and the dominant backgrounds, mainly \(\ttbar\)+jets and \(Z\to\tau\tau\).  

The results obtained show that, although the sensitivity is still limited by the available statistics, the standalone analysis of the \thqb + \ttH processes already demonstrates the feasibility of probing these production modes in the fully hadronic \(\tau\tau\) final state. In the inclusive \(\ttH\) measurement, without splitting into STXS bins, a reduction in the uncertainty of the signal strength has been achieved compared to previous iterations. Particular attention has also been devoted to the NFs constraining the main background processes, which play a crucial role in a standalone fit where no additional Higgs decay channels are included, unlike in the global \(\htautau\) analysis.  

It must be emphasised that systematic uncertainties have not yet been included at this stage. Their incorporation, and the study of their impact on the final measurement, is ongoing at the time of this thesis. An Asimov fit including the full set of systematics is currently being prepared; once their effect on the sensitivity has been evaluated, the analysis will proceed to the unblinding step and a first comparison with data will become possible.  

In summary, this study establishes the methodology and baseline sensitivity for probing \(tHqb\) production in a novel \(\tau_{\mathrm{had}}\tau_{\mathrm{had}}\) topology, in combination with \ttH. While the current results should be regarded as preliminary, the approach represents a significant step forward. With the inclusion of systematic uncertainties and the accumulation of larger Run-3 datasets, the analysis will provide more precise measurements and contribute to extending the reach of direct probes of the top Yukawa coupling, complementing the established evidence from \ttH.  
