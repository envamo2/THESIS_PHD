The work developed in this thesis illustrates different stages of the ATLAS experiment workflow, from the development of improved identification algorithms for basic physics objects to the implementation of advanced analyses targeting Higgs production processes directly sensitive to the top-quark Yukawa coupling. In this way, the topics addressed provide a broad picture of the multiple layers of experimental effort required to achieve high-precision measurements and extend the sensitivity of the LHC physics programme.  

On the electron performance side, the transition from the likelihood-based method to a deep neural network has provided more powerful discrimination between signal and background electrons, ensuring that subsequent analyses can rely on high-quality inputs. On the physics-analysis side, two directions have been pursued: the study of \(\ttH(\tau\tau)\) in the fully hadronic channel using the full Run-2 dataset, embedded within the global \(\htautau\) analysis, and the combined study of \thqb and \(\ttH\) production with Run-2 plus partial Run-3 data (2022–2024). 

The following sections summarise the main conclusions of these two research lines, highlighting the impact of the results achieved within the scope of Run-2 and the ongoing Run-3 programme.

\section{Electron identification using a DNN}

The first major contribution of this thesis has been the development and implementation of a deep neural network (DNN) for electron identification in ATLAS. This represents a methodological shift with respect to the likelihood-based (LH) approach employed during Run-2, introducing a more flexible and powerful discriminant capable of capturing non-linear correlations between input variables.  

The results obtained demonstrate clear gains with respect to the LH method across the full phase space. For electrons in the central barrel region, the DNN improves background rejection by 30–40\% at fixed signal efficiency, particularly against photon conversions and jets misidentified as electrons. In the crack regions, where the LH shows reduced performance, the DNN yields even larger gains, with rejection factors up to 50\% higher in the range 
$1.37 < \eta <1.52$. The performance is also stable across transverse momentum, with the most significant improvements observed at low and intermediate \et ($20 <$ \et $< 40$~GeV).  

At working points corresponding to 80\% signal efficiency, the rejection of heavy-flavour jets is increased by approximately 25\%, while the rejection of light-flavour jets and photon conversions improves by 30–40\%. These gains translate into more robust electron selections for analyses where background suppression is critical.  

Beyond its direct performance improvements, the DNN also provides a more flexible framework for defining working points. A clear example is the treatment of CF electrons: in the LH approach their rejection relied on external discriminants, no longer available in release 22. With the DNN, however, the main discriminant combined with a dedicated CF discriminant recovers and even surpasses the performance of the LH+ECID strategy. This approach can be extended to other background classes, enabling further gains while simplifying the selection.  

In summary, the transition from the LH to the DNN represents a significant step forward in electron identification in ATLAS. The observed improvements in background rejection directly benefit the sensitivity of precision measurements and searches relying on clean electron signatures. The next steps will involve validation in Run-3 data, the derivation of scale factors in control samples, and the integration of the DNN as the default electron identification algorithm in ATLAS analyses.  

\section{\ttH measurement with \htautau at $\sqrt{s}=13$~TeV}

The second contribution of this thesis has been the study of Higgs production in association with a top–antitop pair, with the Higgs decaying to \(\tau\)-leptons in the fully hadronic channel. Within ATLAS, the \(\ttH(\tau\tau)\) channel had already been studied in an earlier Run-2 analysis, although as a ``proof of concept'' with limited sensitivity and relatively large uncertainties. In the most recent analysis presented here, based on the full Run-2 dataset, the strategy was improved by employing more refined multivariate techniques and by extending the measurement to a larger number of STXS bins. The work presented in this thesis has contributed to this development, in particular to the optimisation of the event categorisation and of the discriminants used to separate signal from the dominant backgrounds. A dedicated multiclass BDT training against \(\,Z\to\tau\tau\,+\)jets and \(\ttbar\) was carried out, leading to a gain in sensitivity in the extraction of the \(\ttH\) signal strength in the \(\tau_{\mathrm{had}}\tau_{\mathrm{had}}\) final state.  

The measured cross-section in the \(\tau\tau\) channel, was consistent with the Standard Model expectation within uncertainties. The relative uncertainty was reduced by nearly 18\%, representing a notable improvement in precision. Furthermore, for the first time in this final state, results were extracted in three dedicated \(p_{\mathrm{T}}^{H}\) STXS bins. Due to limited statistical power, upper limits were also derived in these points of interest, which represent the first ATLAS constraints on \(\ttH(\tau\tau)\) production in the STXS framework since the \ttH evidence paper.  

When compared to CMS, which also reported results in this channel using the full Run-2 dataset, the ATLAS analysis achieved competitive sensitivity, with both experiments obtaining measurements compatible with the Standard Model within uncertainties.  

In summary, the \(\ttH(\tau\tau)\) analysis with Run-2 data has demonstrated that, despite the limited statistics available, it is possible to achieve improved sensitivity compared to previous iterations. The results are consistent with the Standard Model and competitive with CMS. Further improvements are expected with additional data from Run-3, enabling more precise measurements. A new ATLAS combination of this analysis with other Higgs decay channels and production modes is currently in preparation, which will allow the impact of the \(\ttH(\tau\tau)\) channel on reducing correlations with other final states to be assessed. In this preliminary combination, including the results presented here, the (anti)correlation between \(\ttH(\tau\tau)\) and \(\ttH(WW)\) is only mildly reduced to about 30\%, showing that there is still considerable room for further improvement.

\section{Study of \ttH + \thqb with \htautau decay at $\sqrt{s}=13$, $13.6$~TeV}

The combined production modes \ttH and \thqb provide direct sensitivity to the $CP$ structure of the Higgs–top interaction. In the Standard Model the coupling is purely $CP$-even, but scenarios with scalar–pseudoscalar mixing remain compatible with current data and would imply new sources of $CP$ violation. Measuring both processes together in the same fit provides sensitivity to potential deviations from the pure $CP$-even hypothesis.  

In this thesis, such a combined analysis has been performed in the fully hadronic \htautau channel, using the full Run-2 dataset together with partial Run-3 data recorded to date (2022–2024). This constitutes the first exploration of \thqb in this topology within ATLAS. The study required a dedicated categorisation strategy based on jet and $b$-tag multiplicities, together with updated object identification algorithms and the development of a new multiclass BDT discriminant to separate the signals from the dominant backgrounds, mainly \ttbar+jets and $Z\to\tau\tau$.  

The results show that, although sensitivity is still limited by the available statistics, the standalone analysis of the \(\ttH+tHqb\) processes already demonstrates the strong potential for measuring both production modes in this decay channel.  
 In the inclusive \(\ttH\) measurement, without splitting into STXS bins, the signal strength uncertainty is reduced compared to the previous iteration of the analysis by nearly 40\%. The combined Asimov fit yields an expected upper limit at 95\% confidence level on the \(tHqb\) cross section of about 10 times the SM prediction, to be compared with the most recent CMS result of \(\mu_{tH} < 14.6\) and the latest ATLAS $tH$ search which obtained \(\mu_{tH} < 13.9\) at 95\% CL. This places the present study in competitive position despite its focus on a single, hadronic final state.  

Only statistical uncertainties have been considered so far, although the impact of systematics is expected to be small. Their incorporation, and the study of their impact on the final measurement, is ongoing at the time of finishing this thesis. An Asimov fit including the full set of systematics is in preparation; once their effect on the sensitivity has been evaluated, the analysis will proceed to the unblinding step and a first comparison with data will become possible.  

In summary, this study establishes the methodology and baseline sensitivity for probing \(tHqb\) production in a novel fully hadronic \htautau topology, in combination with \(\ttH\). While the current results remain preliminary, the approach represents a significant step forward. With the inclusion of systematics and larger Run-3 datasets, the analysis will contribute to extending the reach of direct probes of the top Yukawa coupling, and to testing the $CP$ nature of the Higgs–top interaction, complementing the established evidence from \(\ttH\).  
