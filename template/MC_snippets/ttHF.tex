\subsection[\ttHF]{\ttHF}
\label{subsec:ttHF}
In the following subsections, the set-ups of the current baseline samples for the production of \ttbar quark pairs in association with
$b$-quarks (\ttHF) are described. NLO predictions with massive $b$-quarks in the matrix element and matched to parton shower
programs are available within the \SHERPAOL, \MGNLO and, more recently, \POWHEGBOX frameworks.

\subsubsection[Sherpa]{\SHERPA}
%\label{subsubsec:ttHF_sherpa}
The descriptions below refer to the \SHERPA[2.2.1] samples. 
Details of the set-up are given in Ref.~\cite{ATL-PHYS-PUB-2016-016} and reported below. 

\paragraph{Samples} 
%\label{par:ttHF_sherpa_samples}

The descriptions below correspond to the samples in Table~\ref{tab:ttHF_Sh}.

\begin{table}[htbp]
\begin{center}
\caption{Nominal $t\bar{t}$+HF samples produced with \SHERPA. Variation samples are not explicitly listed.}
\label{tab:ttHF_Sh}
\begin{tabular}{ l | l }
\hline
DSID range & Description \\
\hline
410323--4 & $t\bar{t}$ single lepton \\
410325 & $t\bar{t}$ dilepton \\
410369 & $t\bar{t}$ all-hadronic \\
\hline
\end{tabular}
\end{center}
\end{table}


\paragraph{Description:}

Samples for \ttHF processes were produced with the \SHERPA[2.2.1]~\cite{Bothmann:2019yzt} generator,
using the MEPS@NLO prescription~\cite{Hoeche:2012yf} and interfaced with \OPENLOOPS~\cite{Buccioni:2019sur,Cascioli:2011va,Denner:2016kdg}
to provide the virtual corrections for matrix elements at NLO accuracy.
The four-flavour scheme is used with the $b$-quark mass set to 4.75\,\GeV.  
The renormalisation scale \muR has the functional form 
$\sqrt[4]{m_\text{T}(t) \cdot m_\text{T}(\bar{t}) \cdot m_\text{T}(b) \cdot m_\text{T}(\bar{b})}$. The 
factorisation scale \muF was set to $H_\text{T}/2$, where $H_\text{T}$ is the transverse-mass sum of the
partons in the matrix element, and this value was also the resummation scale \muQ of the parton shower.
The \CT[10nlo] PDF set was used in conjunction with a dedicated PS tune developed by the \SHERPA authors. 

\subsubsection[MadGraph5\_aMC@NLO+Pythia8]{\MGNLOPY[8]}
%\label{subsubsec:ttHF_aMCP8}
In the following, set-ups are described for \PYTHIA only. 
Details of the set-up are given in Ref.~\cite{ATL-PHYS-PUB-2016-016} and reported below. 

\paragraph{Samples}
%\label{par:ttHF_aMCP8_samples}

The descriptions below correspond to the samples in Table~\ref{tab:ttHF_amc}.
\begin{table}[htbp]
\begin{center}
\caption{Nominal $t\bar{t}$+HF samples produced with \MGNLOPY[8]. } 
\label{tab:ttHF_amc}
\begin{tabular}{ l | l }
\hline
DSID range & Description \\
\hline
410265 & $t\bar{t}$ non-all-hadronic \\
410266 & $t\bar{t}$ dileptonic \\
410267 & $t\bar{t}$ all-hadronic \\
\hline
\end{tabular}
\end{center}
\end{table}

\paragraph{Description:}

Samples for \ttHF processes were produced with the \MGNLO generator 
with the \NNPDF[3.0nlo]~\cite{Ball:2014uwa} PDF set. It was interfaced with \PYTHIA[8.230]~\cite{Sjostrand:2014zea},
using the A14 set of tuned parameters~\cite{ATL-PHYS-PUB-2014-021} and the \NNPDF[2.3lo] PDF.
The four-flavour scheme was used with the $b$-quark mass set to 4.75\,\GeV.  
The renormalisation scale \muR has the functional form 
$\sqrt[4]{m_\text{T}(t) \cdot m_\text{T}(\bar{t}) \cdot m_\text{T}(b) \cdot m_\text{T}(\bar{b})}$. The 
factorisation scale \muF was set to $H_\text{T}/2$, where $H_\text{T}$ is the transverse-mass sum of the partons in the matrix
element.
The resummation scale \muQ has the form $\muQ = f_\text{Q} \sqrt{\hat{s}}$, where 
the prefactor $f_\text{Q}$ is an external parameter randomly distributed in the 
range $[f^\text{min}_\text{Q}$, $f^\text{max}_\text{Q}]=[0.1,0.25]$.    

\subsubsection[PowhegBoxRes+Pythia8]{\POWHEGBOXRES+\PYTHIA[8]}
%\label{subsubsec:ttHF_PP8}
In the following, set-ups are described for \PYTHIA[8] only. 

\paragraph{Samples}
%\label{par:ttHF_PP8_samples}
The descriptions below correspond to the samples in Table~\ref{tab:ttHF_pp8}.
\begin{table}[htbp]
\begin{center}
\caption{Nominal $t\bar{t}$+HF samples produced with \POWHEGBOXRES+\PYTHIA[8]. } 
\label{tab:ttHF_pp8}
\begin{tabular}{ l | l }
\hline
DSID range & Description \\
\hline
411179--80 & $t\bar{t}$ non-all-hadronic \\
411178    & $t\bar{t}$ dileptonic \\
411275    & $t\bar{t}$ all-hadronic \\
\hline
\end{tabular}
\end{center}
\end{table}

\paragraph{Description:}

Samples for \ttHF processes were produced with the \POWHEGBOXRES~\cite{Jezo:2018yaf}
generator and \OPENLOOPS~\cite{Buccioni:2019sur,Cascioli:2011va,Denner:2016kdg}, using a pre-release 
of the implementation of this process in \POWHEGBOXRES provided by the authors~\cite{ttbbPowheg}, 
with the \NNPDF[3.0nlo]~\cite{Ball:2014uwa} PDF set. It was interfaced with \PYTHIA[8.240]~\cite{Sjostrand:2014zea},
using the A14 set of tuned parameters~\cite{ATL-PHYS-PUB-2014-021} and the \NNPDF[2.3lo] PDF set.
The four-flavour scheme was used with the $b$-quark mass set to 4.95\,\GeV.
The factorisation scale was set to $0.5\times\Sigma_{i=t,\bar{t},b,\bar{b},j}m_{\mathrm{T},i}$,
the renormalisation scale was set to $\sqrt[4]{m_{\text{T}}(t)\cdot m_{\text{T}}(\bar{t})\cdot m_{\text{T}}(b)\cdot m_{\text{T}}(\bar{b})}$,
and the \hdamp parameter was set to $0.5\times\Sigma_{i=t,\bar{t},b,\bar{b}}m_{\mathrm{T},i}$.
